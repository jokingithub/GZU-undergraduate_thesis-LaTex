
\pagestyle{originality}
\topskip=0pt

% 圆形数字编号定义
\newcommand{\circled}[2][]{\tikz[baseline=(char.base)]
  {\node[shape = circle, draw, inner sep = 1pt]
  (char) {\phantom{\ifblank{#1}{#2}{#1}}};
  \node at (char.center) {\makebox[0pt][c]{#2}};}}
\robustify{\circled}

% 设置行间距
\setlength{\parskip}{0.4em}
\renewcommand{\baselinestretch}{1.41}

% 顶部空白
\vspace*{-6mm}

% 原创性声明部分
\begin{center}
  \heiti\zihao{2}\textbf{原创性声明}
\end{center}

% 本部分字号为小三
\zihao{-3}

本人郑重声明:所呈交的毕业设计(论文),是本人在指导老师的指导下独立进行研究所取得的成果。除文中已经注明引用的内容外,本文不包含任何其他个人或集体已经发表或撰写过的研究成果。对本文的研究做出重要贡献的个人和集体,均已在文中以明确方式标明。

特此申明。

\vspace{13mm}

\begin{flushright}
  本人签名:\hspace{40mm}日\hspace{2.5mm}期:\hspace{13mm}年\hspace{8mm}月\hspace{8mm}日
\end{flushright}

\vspace{17mm}

% 使用授权声明部分
\begin{center}
  \heiti\zihao{2}\textbf{关于使用授权的声明}
\end{center}

本人完全了解北京理工大学有关保管、使用毕业设计(论文)的规定,其中包括:\circled{1}学校有权保管、并向有关部门送交本毕业设计(论文)的原件与复印件;\circled{2}学校可以采用影印、缩印或其它复制手段复制并保存本毕业设计(论文);\circled{3}学校可允许本毕业设计(论文)被查阅或借阅;\circled{4}学校可以学术交流为目的,复制赠送和交换本毕业设计(论文);\circled{5}学校可以公布本毕业设计(论文)的全部或部分内容。

\vspace*{1mm}

\begin{flushright}
  \begin{spacing}{1.65}
    \zihao{-3}
    本人签名:\hspace{40mm}日\hspace{2.5mm}期:\hspace{13mm}年\hspace{8mm}月\hspace{8mm}日\\
    指导老师签名:\hspace{40mm}日\hspace{2.5mm}期:\hspace{13mm}年\hspace{8mm}月\hspace{8mm}日
  \end{spacing}
\end{flushright}

\newpage
